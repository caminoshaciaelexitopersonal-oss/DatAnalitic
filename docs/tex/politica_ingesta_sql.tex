\documentclass{article}
\usepackage{geometry}
\geometry{a4paper, margin=1in}
\usepackage{hyperref}

\title{Política de Ingesta de Archivos SQL}
\author{Jules}
\date{\today}

\begin{document}

\maketitle

\section*{Metadatos del Documento}
\begin{itemize}
    \item \textbf{Fecha:} \today
    \item \textbf{Responsable:} Jules
    \item \textbf{Módulo Asociado:} \texttt{backend/mpa/etl/service.py}
    \item \textbf{Propósito:} Definir las reglas de negocio y el comportamiento esperado del \texttt{EtlService} al procesar archivos con extensión \texttt{.sql}.
    \item \textbf{Impacto en el Sistema:} Define el comportamiento funcional de una parte del pipeline ETL.
\end{itemize}

\section{Reglas de Procesamiento}

\begin{enumerate}
    \item \textbf{Sentencias Permitidas:} El servicio solo procesará archivos \texttt{.sql} que contengan sentencias \texttt{SELECT}. Cualquier otra sentencia (e.g., \texttt{INSERT}, \texttt{UPDATE}, \texttt{CREATE}) será ejecutada pero no devolverá un DataFrame.

    \item \textbf{Manejo de Múltiples Sentencias:} Si un archivo contiene múltiples sentencias, se ejecutarán en orden. Sin embargo, solo el resultado de la \textbf{última sentencia \texttt{SELECT}} será retornado como un DataFrame.

    \item \textbf{Manejo de Errores:} Si el archivo contiene SQL inválido que produce un error durante la ejecución, el servicio capturará la excepción, registrará un error y retornará un DataFrame vacío o \texttt{None}. No detendrá el pipeline completo.

    \item \textbf{Resultados Vacíos:} Si una consulta \texttt{SELECT} se ejecuta correctamente pero no devuelve filas, el servicio retornará un DataFrame vacío con las columnas correspondientes.

    \item \textbf{Límites de Filas:} No se aplicará un límite estricto de filas a nivel de la ingesta de SQL, pero el rendimiento dependerá de la capacidad del motor de base de datos subyacente (SQLite en el caso de las pruebas).
\end{enumerate}

\end{document}
